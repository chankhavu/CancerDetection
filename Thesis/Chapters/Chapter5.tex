% Chapter Template

\chapter{Висновок} % Main chapter title

\label{Chapter5} % Change X to a consecutive number; for referencing this chapter elsewhere, use \ref{ChapterX}

%----------------------------------------------------------------------------------------
%	SECTION 1
%----------------------------------------------------------------------------------------

У цій роботі було використано методи комп'ютерного зору для сегментації зображень інтерфазних ядер букального епітелію, досліджено взаємозв'язок між характеристиками зображень цих ядер та наявністю пухлин в молочній залозі пацієнтів за допомогою фрактальних показників і статистичних методів та запропоновано метод класифікації пацієнтів хворих на рак та на фіброаденоматоз за цими зображеннями на основі новітніх методів машинного навчання. Використаний метод сегментації зображень виявився дуже точним для цієї конкретно поставленої задачі, а запропонований швидкий алгоритм для фільтру зображень на основі еліпсоїдів Петуніна є не менш ефективним за часом, ніж звичайний алгоритм для медіанної фільтрації. Отримано дужі точні результати розпізнавання здорових пацієнтів від хворих раком молочної залози чи фіброаденома- тозом. Запропонований метод класифікації пацієнтів хворих на рак та на фіброаденома- тоз за допомогою глибинного навчання та статистичних критеріїв дає дуже перспективні результати. 



\section{Поради для подальших досліджень}

Відмінність між локальними ознаками на зображеннях інтерфазних ядер букального епітелію у пацієнтів з раком молочної залози та фіброаденоматозом, на відміну від зв'язку між характеристиками цих ядер та наявністю пухлин в молочной залозі пацієнтів \citep{bib:the_beginning}, є ще мало дослідженою. Подальші дослідження у цьому напрямку є дуже перспек- тивними. Автор цієї роботи робить припущення, що використовуючи більш ефективні методи глибинного навчання, можна отримати результати, що будуть набагато точні- шими за отримані.


Як зазначено у розділі \ref{Chapter4}, використання відносно неглибоких архітектур у цій роботі зумовлене відсутністю доступу до більш потужних обчислювальних ресурсів на момент написання цієї роботи. Більш того, новітні дослідження в області глибинного навчання пропонують методи для значного прискорення процесу тренування згорткової нейронної мережі. Так, наприклад, у роботах \citep{nn:perforated} та \citep{nn:acceleration} пропонуються методи прискорення нейрон- них мереж при обчисленнях та навчанні у декілька разів без значної втрати точності. Такі прискорення є доречними при використанні для таких глибоких архітектур як VGG-16 \citep{nn:vgg}, AlexNet \citep{nn:krizhevsky_imagenet} та NIN \citep{nn:nin}. Рекомендується також обрати більш оптимальні гіперпараметри, наприклад, початкові ваги чи коефіцієнти для RMSProp чи інших алго- ритмів оновлення параметрів, а також використовувати методи навчання з вагами \citep{nn:imbalance} чи інші методи для боротьби з проблемою незбалансованості вибірки для тренування. Крім цього, щоб отримати більш точні результати, необхідний більший обсяг даних з кращою якістю для аналізу.