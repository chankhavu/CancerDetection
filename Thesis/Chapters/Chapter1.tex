% Chapter Template

\chapter{Вступ} % Main chapter title

\label{Chapter1} % Change X to a consecutive number; for referencing this chapter elsewhere, use \ref{ChapterX}

Незважаючи на стрімкий розвиток сучасної онкології, захворюваність на рак молочної залози як в Україні, так і в більшості розвинених країн світу продовжує зростати. Щороку у світі реєструють більше п’яти тисяч нових випадків захворювання на рак молочної залози, що становить понад 25\% всіх ракових захворювань у жінок. Тому задача неінвазивної діагностики раку молочної залози є надзвичайно актуальною.

Обробка цифрових зображень давно стала складовою досліджень практично у всіх галу- зях науки. Робота зі зразками з будь-якої предметної області має на увазі не тільки витяг даних із зображень, але і класифікацію знімків, роботу зі складноструктурованими зразками, з неочевидними закономірностями і особливостями, часто помітними лише фахівцям у цій галузі. У медицині можливість автоматично обробляти та розпізнавати великі набори знімків мікроскопа певної тематики може значно вплинути на хід дослід- жень, полегшити процес роботи із зображеннями та у результаті, наприклад, прискорити виявлення хвороби та постановку діагнозу, що допоможе підібрати своєчасне і необхідне лікування.


\section{Метод дослідження}

Як показано у попередніх роботах в області онкології \citep{bib:the_beginning}, злоякісні та доброякісні утво- рення певним чином впливають на інтерфазні ядра букального епітелію. Ми сподіваємось що використовуючи методи комп'ютерного зору, фрактального аналізу та машинного навчання, можна діагнозувати рак молочної залози та фіброаденоматоз (доброякісна пухлина) з певною точністю.

\section{Опис даних}

Для дослідження було використано набір зображень інтерфазних ядер букального епі- телію, отриманих від здорових пацієнтів, а також хворих раком молочної залози та фіброаденоматозом, які лікувались у інституті онкології. Вхідний набір даних для скла- дається з знімків 6752 інтерфазних ядер букального епітелію, для кожного було зроблено 3 знімки мікроскопу (сканограми оптичної щільності): без фільтру, через жовтий фільтр та через пурпурного фільтру (отже всьго 20256 фотографії), взятого з 130 пацієнтів, з них 68 хворих раком, 29 здорових та 33 хворих фіброаденоматозом. 
